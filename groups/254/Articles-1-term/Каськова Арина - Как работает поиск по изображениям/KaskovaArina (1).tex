\documentclass[a4paper,14pt]{extarticle}
\usepackage[T2A]{fontenc}
\usepackage[utf8]{inputenc}
\usepackage[russian]{babel}
\usepackage{geometry}
\geometry{left=25mm, right=25mm, top=25mm, bottom=25mm}
\usepackage{setspace}  % Для управления межстрочными интервалами
\onehalfspacing  % Полуторный интервал

\title{\textbf{Как работает поиск по изображениям?}}
\author{Каськова Арина 10И5}
\date{}

\begin{document}

\maketitle

\section{Введение}
Каждый день люди загружают в интернет огромное количество фотографий, более 3 миллиардов изображений ежедневно. Среди этого обилия найти нужную картинку или понять, что изображено на фотографии - довольно сложная задача для компьютера. Раньше системы сравнивали изображения пиксель в пиксель, но такой подход был неэффективным.Любые небольшие изменения в освещении, ракурсе или кадрировании полностью сбивали работу. Современный поиск по изображениям - это интеллектуальная система, которая не просто сравнивает картинки, а как бы понимает их содержание. В основе этой технологии лежит \textbf{глубокое обучение} и, в частности, \textbf{сверточные нейронные сети}.

\section{Основной принцип работы}
Главная идея современного поиска по изображениям заключается в том, что система не работает с самой картинкой как с набором пикселей. Вместо этого она преобразует изображение в специальный вектор чисел, который называется \textbf{эмбеддингом} (от английского embedding - встраивание). 

Этот вектор компактно описывает смысл изображения. Если на фото кошка, вектор не будет кодировать каждый пиксель по отдельности, а будет находить семантические признаки: четыре лапы, хвост, уши, наличие усов и тд. Две картинки с одинаковым содержанием, например, разные фотографии одной породы кошек будут иметь похожие векторы, даже если они сделаны с разных ракурсов, при разном освещении или с различным фоном. Именно эта деталь делает систему устойчивой и эффективной.

\section{Ключевой алгоритм: сверточная нейронная сеть (CNN)}
Именно сверточные нейронные сети (Convolutional Neural Networks) отвечают за преобразование картинки в её смысловой вектор. Они работают по принципу многослойной фильтрации, имитирующей работу зрительной коры животных:

1. \textbf{Первый слой} ищет простейшие элементы: линии, углы, пятна цвета, градиенты яркости.\\
2. \textbf{Следующие слои} комбинируют эти простые элементы в более сложные: из линий собираются контуры, из контуров - простые формы.\\
3. \textbf{Глубокие слои} распознают целые объекты и их характерные части: морда кошки, лапа собаки, колесо машины, окно в комнате.

Каждый слой - это набор цифровых фильтров (ядер свертки), которые настраиваются в процессе обучения на миллионах размеченных фотографий. Например, чтобы научить сеть распознавать кошек, ей показывают тысячи изображений с пометкой "кошка" и тысячи без кошек. Сеть самостоятельно настраивает свои фильтры, чтобы лучше отличать одно от другого. В итоге последний слой сети выдает конечный вектор-эмбеддинг - уникальный цифровой отпечаток изображения.

\section{Полный цикл работы системы}
Когда пользователь загружает изображение в поисковую систему, происходит четкая последовательность действий:

\textbf{1. Предобработка.} Система приводит изображение к стандартному размеру, обычно 224×224 или 299×299 пикселей, и нормализует цвета. Это важно, потому что нейронная сеть обучена на изображениях определенного формата.

\textbf{2. Извлечение признаков.} Обработанная картинка пропускается через предварительно обученную CNN. На выходе получается вектор, обычно состоящий из 2048 чисел, который и является её цифровым отпечатком.

\textbf{3. Поиск в базе векторов.} Этот вектор сравнивается с миллионами других векторов, заранее вычисленных для всех изображений в базе поисковой системы. Сравнение происходит не по точному совпадению чисел, а по близости в многомерном пространстве. Для этого используют метрики вроде косинусного сходства. Близкие векторы означают семантически похожие изображения.

\textbf{4. Выдача результата.} Система возвращает изображения, чьи векторы оказались ближе всего к вектору конкретного запроса, выстраивая их по степени сходства. Часто также указывается процент уверенности системы в том, что найденные изображения действительно соответствуют запросу.

\section{Почему это не так просто, как кажется}
Несмотря на кажущуюся простоту, технология поиска по изображениям сталкивается с серьезными пунктами:

\begin{itemize}
    \item \textbf{Обучение сети} требует ужасно огромных вычислительных ресурсов и тщательно размеченных данных (миллионы картинок с точными подписями).
    
    \item \textbf{Поиск среди миллиардов векторов} - отдельная сложная задача. Прямой перебор всех векторов при каждом запросе невозможен из-за временных затрат. Поэтому используют оптимизированные алгоритмы приближенного поиска, которые находят не идеально точные, но достаточно хорошие результаты за приемлемо небольшое время.
    
    \item Система должна быть устойчива к каким-либо помехам: разному освещению, ракурсу съемки, наложенному поверх тексту или водяным знакам, частичному перекрытию объекта другими предметами и др.
    
    \item \textbf{Семантическая сложность}: система должна понимать, что фотография кружки и рисунок этой же кружки в мультике - это один и тот же объект, хотя визуально они могут сильно отличаться.
\end{itemize}

\section{Где это применяется?}
Конечно, технология компьютерного зрения, лежащая в основе поиска по изображениям, не остановилась на поиске в интернете и  нашла применение во многих сферах:

\begin{itemize}
    \item \textbf{Медицина:} автоматический анализ МРТ, рентгеновских снимков и результатов микроскопии для помощи в диагностике заболеваний.
    
    \item \textbf{Безопасность:} распознавание лиц в системах видеонаблюдения, автоматическое считывание номеров машин.
    
    \item \textbf{Робототехника:} навигация автономных роботов и дронов, распознавание объектов для манипуляции ими.
    
    \item \textbf{Искусство и дизайн:} поиск плагиата, подбор визуально похожих произведений, рекомендательные системы для фотографий и иллюстраций.
    
    \item \textbf{Торговля:} поиск товаров по фотографии, рекомендации "похожих товаров".
    
    \item \textbf{Сельское хозяйство:} анализ снимков полей с дронов для выявления больных растений или оценки урожайности.
\end{itemize}

\section{Заключение}
Поиск по изображениям - хороший пример того, как сложная математическая модель, сверточная нейронная сеть, решает реальную задачу, используемую по всему миру. От преобразования пикселей в определенные признаки до поиска смысловых соседей в многомерном пространстве. Каждый этап этой технологии представляет собой непростую инженерную и научную проблему. 

Сегодня эта технология продолжает развиваться. Появляются новые трансформеры для компьютерного зрения, улучшаются методы обучения и расширяются области применения.

\end{document}